\documentclass[a4 paper, 15pt]{article}
% Set target color model to RGB
\usepackage[inner=2.0cm,outer=2.0cm,top=2.5cm,bottom=2.5cm]{geometry}
\usepackage{setspace}
\usepackage[rgb]{xcolor}
\usepackage{verbatim}
\usepackage{subcaption, float}
\usepackage{amsgen,amsmath,amstext,amsbsy,amsopn,tikz,amssymb,tkz-linknodes}
\usepackage{fancyhdr, graphicx}
\usepackage[colorlinks=true, urlcolor=blue,  linkcolor=blue, citecolor=blue]{hyperref}
\usepackage[colorinlistoftodos]{todonotes}
\usepackage{rotating}
%\usetikzlibrary{through,backgrounds}
\hypersetup{%
pdfauthor={Ashudeep Singh},%
pdftitle={Homework},%
pdfkeywords={Tikz,latex,bootstrap,uncertaintes},%
pdfcreator={PDFLaTeX},%
pdfproducer={PDFLaTeX},%
}
%\usetikzlibrary{shadows}
\usepackage{booktabs}
\newcommand{\ra}[1]{\renewcommand{\arraystretch}{#1}}
\newcommand{\norm}[1]{\left\lVert#1\right\rVert}

\newtheorem{thm}{Theorem}[section]
\newtheorem{prop}[thm]{Proposition}
\newtheorem{lem}[thm]{Lemma}
\newtheorem{cor}[thm]{Corollary}
\newtheorem{defn}[thm]{Definition}
\newtheorem{rem}[thm]{Remark}
\numberwithin{equation}{section}

\newcommand{\homework}[6]{
   \pagestyle{myheadings}
   \thispagestyle{plain}
   \newpage
   \setcounter{page}{1}
   \noindent
   \begin{center}
   \framebox{
      \vbox{\vspace{2mm}
    \hbox to 6.28in { {\bf CS 6780:~Advanced Machine Learning \hfill {\small (#2)}} }
       \vspace{6mm}
       \hbox to 6.28in { {\Large \hfill #1  \hfill} }
       \vspace{6mm}
       \hbox to 6.28in { {\it Instructor: {\rm #3} \hfill Name: {\rm #5}, Netid: {\rm #6}} }
       %\hbox to 6.28in { {\it TA: #4  \hfill #6}}
      \vspace{2mm}}
   }
   \end{center}
   \markboth{#5 -- #1}{#5 -- #1}
   \vspace*{4mm}
}
\newcommand{\problem}[1]{~\\\fbox{\textbf{Problem #1}}\newline\newline}
\newcommand{\subproblem}[1]{~\newline\textbf{(#1)}}
\newcommand{\D}{\mathcal{D}}
\newcommand{\Hy}{\mathcal{H}}
\newcommand{\VS}{\textrm{VS}}
\newcommand{\solution}{~\newline\textbf{\textit{(Solution)}} }


\begin{document}
\homework{Submission Assignment \#3}{Submitted on: 3/28/2019}{Thorsten Joachims}{}{Molly Ingram, Julien Neves}{msi34, jmn252}

\problem{1}
\subproblem{a}

First, let $\Sigma^d$ be space representing the collection of all possible strings with alphabet $\Sigma$ of lenght $d$.
Since $|\Sigma|= a$, we have that $|\Sigma^d|= a^d$. Additionally, we can enumerate the elements of $\Sigma^d$, in the following way $\Sigma^d = \{s_1,\dots, s_{a^d} \}$.

We can then imagine a new feature space where for $x_i$, characterized by the function $\phi_d(x)$ that maps $x$ to the space $\mathbb{N}^{a^d}$ where $j^th$ component of $\phi_d(x)$ is given by the number of occurences of $s_j$ in $x$.

Then, it is easy to see that $K_d(x_i,x_j)= \phi_d(x_i)\phi_d(x_j)$. Note that the minimum dimensionality of $\phi_d(\cdot)$ is $a^d$, thus as $d$ increases, the feature space increases exponentially in $d$.

\subproblem{b}
\subproblem{c}

It is straightforward to see that for $K(x_i,x_j) = \prod_{d=1}^D K_d(x_i,x_j)$, the implicit feature space of $K(x_i,x_j)$ would simply be the composition of feature spaces of $K_d(x_i,x_j)$ definided in (a), i.e.,
\[
\phi(\cdot) = \phi_1(\cdot) \times \dots \times \phi_D(\cdot)
\]

Since, $\phi_d(\cdot)$ maps into $\mathbb{N}^{a^d}$, we have that $\phi(\cdot)$ would map into $\mathbb{N}^{a^1} \times \dots \times \mathbb{N}^{a^D} = \mathbb{N}^{a+a^2+\dots + a^D}$. Hence, the dimensionality of the feature space of $\phi(\cdot)$ is given by $\sum_{d=1}^D a^d$.


\subproblem{d}

Let $a,b\in \Sigma$ be distinct elements in $\Sigma$. Since $|\Sigma|\geq 2$, $a,b$ exists. Now, we can create the two following strings of length $d$:
\begin{align*}
  u &= \{a,a,\dots, a\}\\
  v &= \{b,b,\dots, b\}
\end{align*}

Then, we can create the following three strings of length $n=2d$ by joining the string $u$ and $v$:
\begin{align*}
  x_1 & = uu\\
  x_2 & = uv\\
  x_3 & = vv
\end{align*}

Then, we have that $K^\prime(x_i,x_j)$ for $x_1,x_2,x_3$ is given by
\begin{center}
\begin{tabular}{ c|c c c }
 & $x_1$ & $x_2$ & $x_3$\\
\hline
$x_1$  & 1 & 1 & 0 \\
$x_2$  & 1 & 1 & 1 \\
$x_3$  & 0 & 1 & 1 \\
 \hline
\end{tabular}
\end{center}
which is not positive semidefinite matrix.

More precisely, if look at the general $K^\prime$ and take $z^T K^\prime z$ where $z=\{1,-1,1,0,0,\dots\}$, i.e., $z=-1$ at the position of $x_2$, $z = 1$ at the position of $x_1$ and $x_3$, and 0 otherwise, then $z^T K^\prime z= -1 <0$.

\newpage
\problem{2}
\subproblem{a}
\subproblem{b}
\subproblem{c}

\newpage
\problem{3}

See Jupyter notebook.

\end{document}
